\section*{Appendix: List library}

Those list library functions that are not built-ins are pre-declared as follows:

\begin{lstlisting}
// is_list recurses down the list and checks that it ends with the empty list null

function is_list(xs) {
    return is_null(xs) || (is_pair(xs) && is_list(tail(xs)));
}

// equal computes the structural equality 
// over its arguments

function equal(x, y){
    return (is_pair(x) && is_pair(y)) 
        ? (equal(head(x), head(y)) &&
           equal(tail(x), tail(y)))
        : (is_null(x) && is_null(y))
          || x === y;
}

// returns the length of a given argument list
// assumes that the argument is a list

function length(xs) {
    return is_null(xs) 
	? 0
        : 1 + length(tail(xs));
}

// map applies first arg f, assumed to be a unary function,
// to the elements of the second argument, assumed to be a list.
// f is applied element-by-element: 
// map(f, [1, [2, null]]) results in [f(1), [f(2), null]]

function map(f, xs) {
    return is_null(xs)
        ? null
        : pair(f(head(xs)), map(f, tail(xs)));
}

// build_list takes a non-negative integer n as first argument,
// and a function fun as second argument.
// build_list returns a list of n elements, that results from 
// applying fun to the numbers from 0 to n-1.

function build_list(n, fun){
    function build(i, fun, already_built) {
	return i < 0
	    ? already_built
	    : build(i - 1, fun, pair(fun(i),
		  		     already_built));
    }
    return build(n - 1, fun, null);
}

// for_each applies first arg fun, assumed to be a unary function,
// to the elements of the second argument, assumed to be a list.
// fun is applied element-by-element:
// for_each(fun, [1, [2, null]]) results in the calls fun(1) and fun(2).
// for_each returns true.

function for_each(fun, xs) {
    if (is_null(xs)) {
	return true;
    } else {
        fun(head(xs));
	return for_each(fun, tail(xs));
    }
}

// to_string uses JavaScript's + to turn its argument into a string

function to_string(x) {
    return x + "";
}

// list_to_string returns a string that represents the argument list.
// It applies itself recursively on the elements of the given list.
// When it encounters a non-list, it applies toString to it.

function list_to_string(xs) {
    return is_null(xs)
        ? "null"
        : is_pair(xs)
            ? "[" + list_to_string(head(xs)) + ","+
                    list_to_string(tail(xs))      + "]"
            : to_string(xs);
}

// reverse reverses the argument, assumed to be a list

function reverse(xs) {
    function rev(original, reversed) {
	return is_null(original)
	    ? reversed
	    : rev(tail(original), 
	          pair(head(original), reversed));
    }
    return rev(xs, null);
}

// append first argument, assumed to be a list, to the second argument.
// In the result null at the end of the first argument list
// is replaced by the second argument, regardless what the second
// argument consists of.

function append(xs, ys) {
    return is_null(xs)
	? ys
        : pair(head(xs),
	       append(tail(xs), ys));
} 

// member looks for a given first-argument element in the 
// second argument, assumed to be a list. It returns the first 
// postfix sublist that starts with the given element. It returns null if the 
// element does not occur in the list

function member(v, xs){
    return is_null(xs)
	? null
        : (v === head(xs))
	    ? xs
	    : member(v, tail(xs));
}

// removes the first occurrence of a given first-argument element
// in second-argument, assmed to be a list. Returns the original 
// list if there is no occurrence.

function remove(v, xs){
    return is_null(xs)
	? null
        : v === head(xs)
	    ? tail(xs)
	    : pair(head(xs), 
		   remove(v, tail(xs)));
}

// Similar to remove, but removes all instances of v
// instead of just the first

function remove_all(v, xs) {
    return is_null(xs)
	? null
        : v === head(xs)
	    ? remove_all(v, tail(xs))
	    : pair(head(xs), 
	  	   remove_all(v, tail(xs)));
}

// filter returns the sublist of elements of the second argument
// (assumed to be a list), for which the given predicate function
// returns true.

function filter(pred, xs){
    return is_null(xs)
	? xs
        : pred(head(xs))
	    ? pair(head(xs),
		   filter(pred, tail(xs)))
	    : filter(pred, tail(xs));
}

// enumerates numbers starting from start, assumed to be a number,
// using a step size of 1, until the number exceeds end, assumed
// to be a number

function enum_list(start, end) {
    return start > end
	? null
        : pair(start,
	       enum_list(start + 1, end));
}

// Returns the item in xs (assumed to be a list) at index n,
// assumed to be a non-negative integer.
// Note: the first item is at position 0

function list_ref(xs, n) {
    return n === 0
	? head(xs)
        : list_ref(tail(xs), n - 1);
}

// accumulate applies an operation op (assumed to be a binary function)
// to elements of sequence (assumed to be a list) in a right-to-left order.
// first apply op to the last element and initial, resulting in r1, then to
// the  second-last element and r1, resulting in r2, etc, and finally
// to the first element and r_n-1, where n is the length of the
// list.
// accumulate(op, zero, list(1, 2, 3)) results in
// op(1, op(2, op(3, zero)))

function accumulate(f, initial, xs) {
    return is_null(xs)
        ? initial
        : f(head(xs),
             accumulate(f, initial, tail(xs)));
}
\end{lstlisting}


