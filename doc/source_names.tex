\section*{Names}

Names\footnote{
In
\href{http://www.ecma-international.org/publications/files/ECMA-ST/Ecma-262.pdf}{
ECMAScript 2016 ($7^{\textrm{th}}$ Edition)},
these names are called \emph{identifiers}.
} start with \verb@_@, \verb@$@ or a
letter\footnote{
By \emph{letter}
we mean \href{http://unicode.org/reports/tr44/}{Unicode} letters (L) or letter numbers (NI).
} and contain only \verb@_@, \verb@$@,
letters or digits\footnote{
By \emph{digit} we mean characters in the
\href{http://unicode.org/reports/tr44/}{Unicode} categories
Nd (including the decimal digits 0, 1, 2, 3, 4, 5, 6, 7, 8, 9), Mn, Mc  and Pc. 
}. Reserved words\footnote{
By \emph{Reserved word} we mean any of:
$\textbf{\texttt{break}}$, $\textbf{\texttt{case}}$, $\textbf{\texttt{catch}}$, $\textbf{\texttt{continue}}$, $\textbf{\texttt{debugger}}$, $\textbf{\texttt{default}}$, $\textbf{\texttt{delete}}$, $\textbf{\texttt{do}}$, $\textbf{\texttt{else}}$, $\textbf{\texttt{finally}}$, $\textbf{\texttt{for}}$, $\textbf{\texttt{function}}$, $\textbf{\texttt{if}}$, $\textbf{\texttt{in}}$, $\textbf{\texttt{instanceof}}$, $\textbf{\texttt{new}}$, $\textbf{\texttt{return}}$, $\textbf{\texttt{switch}}$, $\textbf{\texttt{this}}$, $\textbf{\texttt{throw}}$, $\textbf{\texttt{try}}$, $\textbf{\texttt{typeof}}$, $\textbf{\texttt{var}}$, $\textbf{\texttt{void}}$, $\textbf{\texttt{while}}$, $\textbf{\texttt{with}}$, $\textbf{\texttt{class}}$, $\textbf{\texttt{const}}$, $\textbf{\texttt{enum}}$, $\textbf{\texttt{export}}$, $\textbf{\texttt{extends}}$, $\textbf{\texttt{import}}$, $\textbf{\texttt{super}}$, $\textbf{\texttt{implements}}$, $\textbf{\texttt{interface}}$, $\textbf{\texttt{let}}$, $\textbf{\texttt{package}}$, $\textbf{\texttt{private}}$, $\textbf{\texttt{protected}}$, $\textbf{\texttt{public}}$, $\textbf{\texttt{static}}$, $\textbf{\texttt{yield}}$, $\textbf{\texttt{null}}$, $\textbf{\texttt{true}}$, $\textbf{\texttt{false}}$.
} such as keywords are not allowed as names.

Valid names are \verb@x@, \verb@_45@, \verb@$$@ and $\mathtt{\pi}$,
but always keep in mind that programming is communicating, and therefore the familiarity of the
audience with the characters used in names is an important aspect of program readability.

The following names can be used, in addition to names that
are declared using \texttt{\textbf{const}}, \texttt{\textbf{function}} and
$\texttt{\textbf{=>}}$:
\begin{itemize}
\item \lstinline{math_}$\textit{name}$,
where $\textit{name}$ is any name specified in the
JavaScript
\texttt{Math} library, see\\
\href{https://www.ecma-international.org/ecma-262/8.0/index.html#sec-math-object}{\color{DarkBlue}ECMAScript Specification, Section 20.2}. Examples:
\begin{itemize}
\item \verb#math_PI#: Refers to the mathematical constant $\pi$,
\item \verb#math_sqrt#\texttt{(n)}: Returns the square root of the \emph{number} \texttt{n}.
\end{itemize}
\item \texttt{runtime()}: Returns number of milliseconds elapsed since January 1, 1970 00:00:00 UTC
\item \texttt{display(a)}: Displays \emph{any} value \texttt{a} in the console; returns \texttt{undefined}.
\item \texttt{error(a)}: Displays \emph{any} value \texttt{a} in the console with error flag. The evaluation
  of any call of \texttt{error} aborts the running program immediately.
\item \texttt{prompt(s)}: Pops up a window that displays the \emph{string} \texttt{s}, provides
an input line for the user to enter a text and an ``OK'' button. The call of \texttt{prompt}
suspends execution of the program until the ``OK'' button is pressed, at which point it
returns the entered text as a string.
\item \verb#parse_int#\texttt{(s, i)}:
interprets the \emph{string} \texttt{s} as an integer, using the positive integer \texttt{i} as radix, and returns the respective value,
see \href{https://www.ecma-international.org/ecma-262/8.0/index.html#sec-parseint-string-radix}{\color{DarkBlue}ECMAScript Specification, Section 18.2.5}.
\item \verb#undefined#, \verb#NaN#, \verb#Infinity#: Refer to JavaScript's undefined,
NaN (``Not a Number'') and Infinity values, respectively.
\end{itemize}
